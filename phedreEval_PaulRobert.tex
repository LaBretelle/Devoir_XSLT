 
\documentclass[12pt, a4paper]{report}
\usepackage[utf8x]{inputenc}
\usepackage[T1]{fontenc}
\usepackage{lmodern}
\usepackage{graphicx}
\usepackage[french]{babel}
\usepackage{reledmac}
\usepackage{glossaries}
\makeglossaries
\newglossaryentry{Phedre}{name={Phèdre},description={Reine d'Athènes}}\newglossaryentry{Nourrice}{name={La Nourrice},description={Nourrice et confidente de Phèdre.}}\newglossaryentry{Thésée}{name={Thésée},description={Roi d'Athènes.}}\newglossaryentry{Pluton}{name={Pluton},description={Dieu des enfers.}}\newglossaryentry{Venus}{name={Vénus},description={Déesse de l'amour.}}\newglossaryentry{Athéna}{name={Athéna},description={Déesse de la sagesse.}}\newglossaryentry{Cupidon}{name={Cupidon},description={Dieu de l'amour.}}\newglossaryentry{Apollon}{name={Phébus},description={Dieu incarnant le soleil.}}\newglossaryentry{Antiope}{name={Antiope},description={Reine des Amazones.}}\newglossaryentry{Pirithoüs}{name={Pirithoüs},description={Roi des Lapithes.}}\newglossaryentry{Ariane}{name={Ariane},description={Soeur de Phèdre et fille du roi de Crète, Minos.}}\newglossaryentry{Alcmène}{name={Alcmène},description={Reine de Tirynthe et mère d'Héraclès.}}\newglossaryentry{Styx}{name={Le fleuve Styx},description={Fleuve mythologique des enfers}}\newglossaryentry{Athenes}{name={Pallas-Athènes},description={Pallas-Athena est un des noms de la déesse Athéna, déesse protectrice de la
                     ville d'Athènes}}\newglossaryentry{Hespérie}{name={Espagne},description={Désigne de manière générique les "terres à l'ouest". Du point de vue des latins, c'est l'Espagne, et du point de vue des grecs c'est l'Italie.}}\newglossaryentry{TropiqueCancer}{name={Tropique du Cancer},description={Ces pays ont en commun d'être familiers des romains et des grecs, et d'être
                     traversés par le tropique du Cancer. Cela peut être les pays actuels de l'Algérie, du Niger, de la Mauritanie, de l'Egypte, de la Lybie et de l'Arabie Saoudite.}}\newglossaryentry{Nord}{name={Nord de l'Italie},description={Ces régions sont froides, et souvent peuplées de nomades.}}\newglossaryentry{Thessalie}{name={Thessalie},description={Région au sud de la Macédoine et au nord de l'Epire faisant partie de la Grèce antique.}}\newglossaryentry{Perse}{name={Perse},description={Pays recouvrant notammant l'Iran.}}\newglossaryentry{Lydie}{name={Lydie},description={Pays de l'Antiquité situé en Asie Mineure, à l'est de l'Eolie.}}\newglossaryentry{Tyr}{name={Tyr},description={Ville de Phénicie, au sud de l'actuel Liban.}}
\setstanzaindents{0,1}
\setcounter{stanzaindentsrepetition}{1}        
            
\Xarrangement[A]{paragraph}
\Xparafootsep{$\parallel$~}

\title{Phèdre \\ \textit{Phaedra}}\author{Sénèque\\ \textit{Lucius Annaeus Seneca}}

\begin{document}

\maketitle

\firstlinenum{5}
\linenumincrement{5}
\linenummargin{right}
\chapter{texte}    
\beginnumbering
\stanza 
Amoris in me maximum regnum \edtext{puto}{\Afootnote{fero  A, reor  Z}} & reditusque nullos metuo: non umquam amplius & convexa tetigit supera qui mersus semel  & adiit silentem nocte perpetua domum. & Ne crede \edtext{Diti}{\Afootnote{ditis  A}}. Clauserit regnum licet & canisque diras \gls{Styx} observet fores: & solus negatas invenit \gls{Thésée} vias. & Veniam ille amori forsitan nostro dabit.  & Immitis etiam coniugi castae fuit: & experta \edtext{saevam est}{\Afootnote{est saevam  A}} barbara \gls{Antiope} manum. & sed posse flecti coniugem iratum puta: & quis huius animum flectet intractabilem? & exosus omne feminae nomen fugit,  & immitis annos caelibi vitae dicat, & conubia vitat: genus Amazonium scias. & Hunc in nivosi collis haerentem iugis, & et aspera agili saxa calcantem pede & sequi per alta nemora, per montes placet.  & Resistet ille seque mulcendum dabit & castosque ritus \gls{Venus} non
                           casta exuet? & Resistet ille seque mulcendum dabit & castosque ritus \gls{Venus} non
                              casta exuet? & tibi ponet odium, cuius odio forsitan & persequitur omnes? precibus haud vinci potest. & tibi ponet odium, cuius odio forsitan & persequitur omnes? precibus haud vinci potest. & tibi ponet odium, cuius odio forsitan & persequitur omnes? precibus aut vinci potest. & tibi ponet odium, cuius odio forsitan & persequitur omnes? precibus haud vinci potest. & \edtext{ Ferus est? amore didicimus vinci feros. }{\Afootnote{Ferus est? Amore didicimus vinci feros.  A}} & Fugiet. & Per ipsa maria \edtext{si fugiet}{\Afootnote{si fugiat  A}}, sequar. & Patris memento. & Meminimus matris simul. & Genus omne profugit. & Paelicis careo metu. & Aderit maritus— & Nempe \gls{Pirithoüs} comes? & Aderitque genitor, & Mitis, \gls{Ariane} pater.  & Per has senectae \edtext{splendidas}{\Afootnote{splendida  E}} supplex comas & fessumque curis pectus et cara ubera & precor, furorem siste teque \edtext{ipsa}{\Afootnote{  A}} adiuva: & \edtext{pars}{\Afootnote{  E}} sanitatis velle sanari fuit. & Non omnis animo cessit \edtext{ingenuo}{\Afootnote{ingenio  E}} pudor.  & paremus, altrix. qui regi non vult amor & vincatur. haud te, fama, maculari sinam. & haec sola ratio est, unicum effugium mali: & virum sequamur, morte praevertam nefas. & Moderare, alumna, \edtext{mentis}{\Afootnote{mortis  A}} effrenae impetus,  & animos coerce, dignam ob hoc \edtext{vita}{\Afootnote{ultra  E}} reor & quod esse temet autumas dignam nece. & Decreta mors est: quaeritur fati genus. & laqueone vitam finiam an ferro incubem? & an missa praeceps\gls{Athenes} cadam?  & \edtext{Proin castitatis vindicem armemus manum.}{\Afootnote{pro(h) castitatis vindicem armemus manum.  A, \textit{omisit}  G}} & \edtext{Sic}{\Afootnote{si  A}} te senectus nostra praecipiti sinat  & perire leto? \edtext{siste}{\Afootnote{sisteque  S, sique  P  T  V}} furibundum impetum. & \edtext{ haud quisquam ad vitam facile revocari potest }{\Afootnote{\textit{omisit}  A}} & Prohibere nulla ratio periturum potest:  & ubi \edtext{qui}{\Afootnote{quis  A}}mori constituit et debet mori. & \edtext{}{\Afootnote{
                        Proin
                        pro(h)
                      castitatis vindicem armemus manum.  G}} & Solamen annis unicum fessis, era,  & si tam protervus incubat menti furor, & contemne famam— fama vix vero favet, & peius merenti melior et peior bono.  & temptemus animum tristem et intractabilem. & meus iste labor est aggredi iuvenem ferum & mentemque saevam flectere immitis viri. & Diva non miti generata ponto, & quam vocat matrem geminus \gls{Cupidon},  &  impotens flammis simul et sagittis  & iste \edtext{lascivus puer et renidens}{\Afootnote{puer lascivus et acre nitens  A}} & tela quam certo \edtext{moderatur}{\Afootnote{jaculatur  A}} arcu! & \edtext{ labitur totas furor in medullas }{\Afootnote{\textit{omisit}  A}} & \edtext{igne furtivo populante venas.}{\Afootnote{\textit{omisit}  A}} & non habet latam data plaga frontem, & sed \edtext{vorat}{\Afootnote{vocat  A}} tectas penitus medullas, & nulla pax isti puero: per orbem & spargit effusas agilis sagittas; & quaeque nascentem videt ora solem,  & quaeque ad \edtext{Hesperiae iacet ora
                     metas}{\Afootnote{occasus jacet pra seros  A}}, & si qua ferventi subiecta \gls{TropiqueCancer}\edtext{}{\Afootnote{est  A  BOTHE  C  E  F  G  GROTIUS  K  L  M  N  Ox  P  Q  R  S  T  Th  V  Z  d  e  ecl  τ  recc  perseus  ed}}, & si qua \edtext{Parrhasiae}{\Afootnote{majoris  A}} glacialis ursae & semper errantes patitur colonos,  & novit hos aestus, iuvenum feroces  & concitat flammas senibusque fessis & rursus extinctos revocat calores, & virginum ignoto ferit igne pectus & et iubet caelo superos relicto & vultibus falsis habitare terras.  & \gls{Thessalie}
               \gls{Apollon} pecoris magister & egit armentum positoque plectro & impari tauros calamo vocavit, & induit formas quotiens minores & ipse qui caelum nebulasque \edtext{ducit}{\Afootnote{fecit  E}}:  & Candidas ales modo movit alas, & dulcior \edtext{vocem}{\Afootnote{voce  E  A}} moriente cygno ; & fronte nunc torva petulans iuvencus & virginum stravit sua terga ludo, & perque fraternos \edtext{nova}{\Afootnote{mala  A}} regna fluctus  & ungula lentos imitante remos & * pectore adverso domuit profundum, & pro sua vector timidus rapina, & arsit obscuri dea clara mundi & nocte deserta nitidosque fratri  & tradidit currus aliter regendos: & ille nocturnas agitare bigas & discit et gyro breviore flecti,  & \edtext{dum tremunt axes graviore curru; }{\Afootnote{\textit{omisit}  A  BOTHE  C  E  F  G  GROTIUS  K  L  M  N  Ox  P  Q  R  S  T  Th  V  Z  d  e  ecl  τ  recc  ed}} & nec suum tempus tenuere nectes  & et dies tardo remeavit ortu.  & \edtext{}{\Afootnote{dum tremunt axes graviore curru;   A  BOTHE  C  E  F  G  GROTIUS  K  L  M  N  Ox  P  Q  R  S  T  Th  V  Z  d  e  ecl  τ  recc  perseus  ed}} & natus \gls{Alcmène} posuit pharetras  & et minax vasti spolium leonis, & passus aptari digitis zmaragdos & et dari legem rudibus capillis;  & crura distincto religavit auro. & luteo plantas cohibente socco; & et manu, clavam modo qua gerebat, & fila deduxit properante fuso: & \edtext{Vidit}{\Afootnote{Vidi  E}}
               \gls{Perse} ditique ferox  & \gls{Lydie}
               \edtext{regno}{\Afootnote{regni  C  recc, harenae  GROTIUS, harena  Z}} & deiecta feri terga leonis & umerisque, quibus sederat alti & regia caeli, & tenuem \gls{Tyr} stamine pallam.  \& 
\endnumbering
\printglossaries
\end{document}
        
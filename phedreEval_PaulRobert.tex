 
\documentclass[12pt, a4paper]{report}
\usepackage[utf8x]{inputenc}
\usepackage[T1]{fontenc}
\usepackage{lmodern}
\usepackage{graphicx}
\usepackage[french]{babel}
\usepackage{reledmac}
\usepackage{glossaries}
\makeglossaries
\newglossaryentry{Phaedra}{name={Phaedra},description={Reine d'Athènes}}\newglossaryentry{Nutrix}{name={Nutrix},description={Nourrice et confidente de Phèdre.}}\newglossaryentry{Theseus}{name={Theseus},description={Roi d'Athènes.}}\newglossaryentry{Diti}{name={Diti},description={Dieu des enfers, aussi appelé Dis et Dis Pater.}}\newglossaryentry{Venere}{name={Venere},description={Déesse de l'amour.}}\newglossaryentry{Palladia}{name={Palladia},description={Déesse de la sagesse.}}\newglossaryentry{Cupido}{name={Cupido},description={Dieu de l'amour.}}\newglossaryentry{Phoebus}{name={Phoebus},description={Dieu incarnant le soleil, aussi appelé Phébus.}}\newglossaryentry{Antiope}{name={Antiope},description={Reine des Amazones.}}\newglossaryentry{Perithoi}{name={Perithoi},description={Roi des Lapithes.}}\newglossaryentry{Ariadnae}{name={Ariadnae},description={Soeur de Phèdre et fille du roi de Crète, Minos.}}\newglossaryentry{Alcmena}{name={Alcmena},description={Reine de Tirynthe et mère d'Héraclès.}}\newglossaryentry{Stygius}{name={Stygius},description={Fleuve mythologique des enfers}}\newglossaryentry{arce}{name={arce},description={Pallas-Athena est un des noms de la déesse Athéna, déesse protectrice de la
                     ville d'Athènes}}\newglossaryentry{Hesperiae}{name={Hesperiae},description={Désigne de manière générique les "terres à l'ouest". Du point de vue des latins, c'est l'Espagne, et du point de vue des grecs c'est l'Italie.}}\newglossaryentry{cancro}{name={cancro},description={Ces pays ont en commun d'être familiers des romains et des grecs, et d'être
                     traversés par le tropique du Cancer. Cela peut être les pays actuels de l'Algérie, du Niger, de la Mauritanie, de l'Egypte, de la Lybie et de l'Arabie Saoudite.}}\newglossaryentry{Parrhasiae}{name={Parrhasiae},description={Ces régions sont froides, et souvent peuplées de nomades.}}\newglossaryentry{Thessali}{name={Thessali},description={Région au sud de la Macédoine et au nord de l'Epire faisant partie de la Grèce antique.}}\newglossaryentry{Persis}{name={Persis},description={Pays recouvrant notammant l'Iran.}}\newglossaryentry{Lydia}{name={Lydia},description={Pays de l'Antiquité situé en Asie Mineure, à l'est de l'Eolie.}}\newglossaryentry{Tyrio}{name={Tyrio},description={Ville de Phénicie, au sud de l'actuel Liban.}} 
\setstanzaindents{0,1}
\setcounter{stanzaindentsrepetition}{1}        
            
\Xarrangement[A]{paragraph}
\Xparafootsep{$\parallel$~}

\title{Phèdre \\ \textit{Phaedra}}\author{Sénèque\\ \textit{Lucius Annaeus Seneca}}\date{1996} 

\begin{document}

\maketitle

\firstlinenum{5}
\linenumincrement{5}
\linenummargin{right}
\chapter{texte : vers 218 au vers 330}  

\setline{218} 
\beginnumbering
\stanza 
Phaedra : Amoris in me maximum regnum \edtext{puto}{\Afootnote{fero  A, reor  Z}} &  \qquad reditusque nullos metuo: non umquam amplius &  \qquad convexa tetigit supera qui mersus semel  &  \qquad adiit silentem nocte perpetua domum. & Nutrix : Ne crede \edtext{Diti}{\Afootnote{ditis  A}}. Clauserit regnum licet &  \qquad canisque diras \gls{Stygius} observet fores: &  \qquad solus negatas invenit \gls{Theseus} vias. & Phaedra : Veniam ille amori forsitan nostro dabit.  & Nutrix : Immitis etiam coniugi castae fuit: &  \qquad experta \edtext{saevam est}{\Afootnote{est saevam  A}} barbara \gls{Antiope} manum. &  \qquad sed posse flecti coniugem iratum puta: &  \qquad quis huius animum flectet intractabilem? &  \qquad exosus omne feminae nomen fugit,  &  \qquad immitis annos caelibi vitae dicat, &  \qquad conubia vitat: genus Amazonium scias. & Phaedra : \edtext{Hunc in nivosi collis haerentem iugis,}{\Afootnote{\textit{omisit}  A}} &  \qquad \edtext{et aspera agili saxa calcantem pede
               }{\Afootnote{\textit{omisit}  A}} &  \qquad \edtext{sequi per alta nemora, per montes placet. 
               }{\Afootnote{\textit{omisit}  A}} &  \qquad 
               \edtext{Resistet ille seque mulcendum dabit}{\Afootnote{Nutrix :   τ  K  Q  e  Ox  d  L  n  r}} &  \qquad \edtext{castosque ritus Venere non
               casta exuet?}{\Afootnote{Nutrix :   τ  K  Q  e  Ox  d  L  n  r}} & Nutrix : tibi ponet odium, cuius odio forsitan &  \qquad \edtext{persequitur omnes? precibus haud vinci potest.}{\Afootnote{Phèdre :   τ  K  Q  e  Ox  d  L  n  r}} & Phaedra : \edtext{ Ferus est? amore didicimus vinci feros. }{\Afootnote{ Nutrix :   A}} & Nutrix : Fugiet. & Phaedra : Per ipsa maria \edtext{si fugiet}{\Afootnote{si fugiat  A}}, sequar. & Nutrix : Patris memento. & Phaedra : Meminimus matris simul. & Nutrix : Genus omne profugit. & Phaedra : Paelicis careo metu. & Nutrix : Aderit maritus— & Phaedra : Nempe \gls{Perithoi} comes? & Nutrix : Aderitque genitor, & Phaedra : Mitis, \gls{Ariadnae} pater.  & Nutrix : Per has senectae \edtext{splendidas}{\Afootnote{splendida  E}} supplex comas &  \qquad fessumque curis pectus et cara ubera &  \qquad precor, furorem siste teque \edtext{ipsa}{\Afootnote{  A}} adiuva: &  \qquad \edtext{pars}{\Afootnote{  E}} sanitatis velle sanari fuit. & Phaedra : Non omnis animo cessit \edtext{ingenuo}{\Afootnote{ingenio  E}} pudor.  &  \qquad paremus, altrix. qui regi non vult amor &  \qquad vincatur. haud te, fama, maculari sinam. &  \qquad haec sola ratio est, unicum effugium mali: &  \qquad virum sequamur, morte praevertam nefas. & Nutrix : Moderare, alumna, \edtext{mentis}{\Afootnote{mortis  A}} effrenae impetus,  &  \qquad animos coerce, dignam ob hoc \edtext{vita}{\Afootnote{ultra  E}} reor &  \qquad quod esse temet autumas dignam nece. & Phaedra : Decreta mors est: quaeritur fati genus. &  \qquad laqueone vitam finiam an ferro incubem? &  \qquad an missa praeceps\gls{arce} \gls{Palladia} cadam?  &  \qquad \edtext{Proin castitatis vindicem armemus manum.}{\Afootnote{pro(h) castitatis vindicem armemus manum.  A, \textit{omisit}  G}} & Nutrix : \edtext{Sic}{\Afootnote{si  A}} te senectus nostra praecipiti sinat  &  \qquad perire leto? \edtext{siste}{\Afootnote{sisteque  S, sique  P  T  V}} furibundum impetum. &  \qquad \edtext{ haud quisquam ad vitam facile revocari potest }{\Afootnote{\textit{omisit}  A}} & Phaedra : Prohibere nulla ratio periturum potest:  &  \qquad ubi \edtext{qui}{\Afootnote{quis  A}}mori constituit et debet mori. &  \qquad \edtext{}{\Afootnote{
                        Proin
                        pro(h)
                      castitatis vindicem armemus manum.  G}} & Nutrix : Solamen annis unicum fessis, era,  &  \qquad si tam protervus incubat menti furor, &  \qquad contemne famam— fama vix vero favet, &  \qquad peius merenti melior et peior bono.  &  \qquad temptemus animum tristem et intractabilem. &  \qquad meus iste labor est aggredi iuvenem ferum &  \qquad mentemque saevam flectere immitis viri. & Chorus : Diva non miti generata ponto, &  \qquad quam vocat matrem geminus \gls{Cupido},  &  \qquad  impotens flammis simul et sagittis  &  \qquad iste \edtext{lascivus puer et renidens}{\Afootnote{puer lascivus et acre nitens  A}} &  \qquad tela quam certo \edtext{moderatur}{\Afootnote{jaculatur  A}} arcu! &  \qquad \edtext{ labitur totas furor in medullas }{\Afootnote{\textit{omisit}  A}} &  \qquad \edtext{igne furtivo populante venas.}{\Afootnote{\textit{omisit}  A}} &  \qquad non habet latam data plaga frontem, &  \qquad sed \edtext{vorat}{\Afootnote{vocat  A}} tectas penitus medullas, &  \qquad nulla pax isti puero: per orbem &  \qquad spargit effusas agilis sagittas; &  \qquad quaeque nascentem videt ora solem,  &  \qquad quaeque ad \edtext{Hesperiae iacet ora
                     metas}{\Afootnote{occasus jacet pra seros  A}}, &  \qquad si qua ferventi subiecta \gls{cancro}\edtext{}{\Afootnote{est  A  BOTHE  C  E  F  G  GROTIUS  K  L  M  N  Ox  P  Q  R  S  T  Th  V  Z  d  e  ecl  τ  recc  perseus  ed}}, &  \qquad si qua \edtext{Parrhasiae}{\Afootnote{majoris  A}} glacialis ursae &  \qquad semper errantes patitur colonos,  &  \qquad novit hos aestus, iuvenum feroces  &  \qquad concitat flammas senibusque fessis &  \qquad rursus extinctos revocat calores, &  \qquad virginum ignoto ferit igne pectus &  \qquad et iubet caelo superos relicto &  \qquad vultibus falsis habitare terras.  &  \qquad \gls{Thessali}
               \gls{Phoebus} pecoris magister &  \qquad egit armentum positoque plectro &  \qquad impari tauros calamo vocavit, &  \qquad induit formas quotiens minores &  \qquad ipse qui caelum nebulasque \edtext{ducit}{\Afootnote{fecit  E}}:  &  \qquad Candidas ales modo movit alas, &  \qquad dulcior \edtext{vocem}{\Afootnote{voce  E  A}} moriente cygno ; &  \qquad fronte nunc torva petulans iuvencus &  \qquad virginum stravit sua terga ludo, &  \qquad perque fraternos \edtext{nova}{\Afootnote{mala  A}} regna fluctus  &  \qquad ungula lentos imitante remos &  \qquad * pectore adverso domuit profundum, &  \qquad pro sua vector timidus rapina, &  \qquad arsit obscuri dea clara mundi &  \qquad nocte deserta nitidosque fratri  &  \qquad tradidit currus aliter regendos: &  \qquad ille nocturnas agitare bigas &  \qquad discit et gyro breviore flecti,  &  \qquad \edtext{dum tremunt axes graviore curru; }{\Afootnote{\textit{omisit}  A  BOTHE  C  E  F  G  GROTIUS  K  L  M  N  Ox  P  Q  R  S  T  Th  V  Z  d  e  ecl  τ  recc  ed}} &  \qquad nec suum tempus tenuere nectes  &  \qquad et dies tardo remeavit ortu.  &  \qquad \edtext{}{\Afootnote{dum tremunt axes graviore curru;   A  BOTHE  C  E  F  G  GROTIUS  K  L  M  N  Ox  P  Q  R  S  T  Th  V  Z  d  e  ecl  τ  recc  perseus  ed}} &  \qquad natus \gls{Alcmena} posuit pharetras  &  \qquad et minax vasti spolium leonis, &  \qquad passus aptari digitis zmaragdos &  \qquad et dari legem rudibus capillis;  &  \qquad crura distincto religavit auro. &  \qquad luteo plantas cohibente socco; &  \qquad et manu, clavam modo qua gerebat, &  \qquad fila deduxit properante fuso: &  \qquad \edtext{Vidit}{\Afootnote{Vidi  E}}
               \gls{Persis} ditique ferox  &  \qquad \gls{Lydia}
               \edtext{regno}{\Afootnote{regni  C  recc, harenae  GROTIUS, harena  Z}} &  \qquad deiecta feri terga leonis &  \qquad umerisque, quibus sederat alti &  \qquad regia caeli, &  \qquad tenuem \gls{Tyrio} stamine pallam.  \&  
\endnumbering
\printglossaries
\end{document}
        